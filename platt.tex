\selectlanguage{ngerman}
\renewcommand{\headline}{Penta$\cdot$speel - Plattdüütsch}
\renewcommand{\tocent}{Plattdüütsch}
\renewcommand{\translator}{J.S.}
\renewcommand{\general}{
In een Minute to begripen aver johrelong to speelen.

Twee Speler broken blot 20 Minuten. Dree oder veer speelen 40-99 Minuten.

Wörpel gifft dat nich. Goot för Speeler af 5 Johren. En Speel vun Jan \noun{Suchanek}.

In de Schachtel: 4$\times$5 farvige Poppen, 5 schwatte un 5 grooge Blöck un de Speelbrett.
}

\renewcommand{\choosext}{Poppen}
\renewcommand{\choosex}{
Elk Speeler föhrt en Trupp vun fief Poppen.

Jedereen sökt sich en so'n Trupp ut. 

In jede Trupp is 'ne blage, 'ne roote, 'ne witte, 'ne grööne un 'ne geele Popp. 

De löpen all vun de fief Ecken to de fief Krüüzen. 
}

\renewcommand{\setupt}{Opbau}
\renewcommand{\setup}{

Stellt de Poppen no Forben ordnet op de fief Ecken op de Krink: de witten op de witte Eck, de blage up de blage, usw. 

Op jedereen Krüüz kümmt en schwatte Block.

De groogen bleeven eerstmol in de Mitt. 
}

\renewcommand{\objectivet}{Teel vun de Speel}
\renewcommand{\objective}{
Witte Poppen wulln to dat witte Krüüz, blage to dat blage, usw. 

Vun de Krink her is de Teel ümmer de Krüüz mit de sülve Farve gegenöver. 

Welkeen toerst \emph{dree} vun sünne Poppen op de rechte Teel treckt, winnt.
}

\renewcommand{\rulest}{Regelwark}
\renewcommand{\rules}{
Treck een vun diene Puppen op de Stiern oder de Krink as de möögst in egal welke Richtung. 

An jedereen Krüüz kannst afbögen ahn anhollen. 

Du dörvst trecken, liekers nich jumpen! Kien över Blöck of över anner Poppen!

\myskip

Aver du kannst op een Feld trecken, dat besett is, un slaan:

\myskip

Slaagst du en swatten Block, sett eem up egal en frieet Feld. 

Slaagst du en anner Popp, denn tuuschen de beeden Poppen de Steel.

\myskip

So kann man ook twee vun sünne egenen Poppen tuuschen.

Treckst du op een Feld, op de noch veele Poppen stahn, muttst mit \emph{een }vun jüm tuuschen. 

\myskip 

Man dörv nich tweeemol de sülven Treck moken. 

\myskip 

Kummt  en Popp to ehr Teel hen, treckt se rut. Se kömmt in de Mitt. Dorför kreegt man vun dor een groogen Block up en frieet Feld to setten. 

Slaagt man so en groogen Block, kümmt he wedder rut.

\myskip

Welkeen toerst \emph{dree }vun sien Poppen uttreckt, winnt. 

\myskip

Sabbel nich, dat geiht!
}
