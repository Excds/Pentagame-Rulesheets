\selectlanguage{russian}
\renewcommand{\headline}{Penta$\cdot$game - Rусский}
\renewcommand{\tocent}{Русский}
\renewcommand{\translator}{alg}
\renewcommand{\general}{
Простая и интересная игра. Автор: Jan \noun{Suchanek}

Игра в среднем длится от 20 до 90 минут, в зависимости от количества игроков.

В коробке: 4$\times$5 разноцветных фигурок, 5 черных и
5 серых фигурок, игровое поле.
}

\renewcommand{\choosext}{Выбор фигурок}
\renewcommand{\choosex}{
Каждый игрок получает пять разноцветных фигурок одного типа.

Фигурки стартуют в вершинах пентаграммы.

Фигурки хотят достигнуть клетки в соответствующего цвета в середине поля.
}
\renewcommand{\setupt}{Подготовка}
\renewcommand{\setup}{

Расставьте фигурки в вершинах пентаграммы в соответствии с их цветом.

Расставьте черные фигурки на цветные клетки на пересечениях линий.

Поставьте серые фигурки в середину поля \cyrdash\  они понадобятся позже.

}

\renewcommand{\objectivet}{Цель}
\renewcommand{\objective}{
Выигрывает первый, кто переместит три фигурки в соответствующие их цвету клетки на пересечениях линий пентаграммы.
}

\renewcommand{\rulest}{Правила}
\renewcommand{\rules}{
В свой ход можно переместить любую свою фигурку на любое расстояние вдоль линий пентаграммы или кольца.

На развилках можно поворачивать без остановки.

Нельзя перепрыгивать через другие фигурки.

\myskip

Если остановиться \emph{на} занятой клетке:

\myskip

Если в клетке черная фигурка \cyrdash\ переместите ее в любую свободную клетку.

Если в клетке фигурка, принадлежащая игроку \cyrdash\ поставьте ее в клетку, в которой начинался ход.
 
\myskip

Если остановиться на клетке с несколькими фигурками \cyrdash\ заместите одну из них.
 
\myskip 
 
Нельзя повторять один и тот же ход дважды.

\myskip 

Фигурка, достигнув назначенной клетки, удаляется с поля.
Игрок, которому принадлежит фигурка, взамен получает серую фигурку, которую можно поставить на любую свободную клетку.

При замещении, серая фигурка удаляется с поля.

\myskip

Во время игры нельзя болтать не по делу.
}
