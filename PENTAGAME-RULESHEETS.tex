\documentclass[12pt]{scrartcl}
\PassOptionsToPackage{english}{babel}
\PassOptionsToPackage{french}{babel}
\PassOptionsToPackage{ngerman}{babel}
\PassOptionsToPackage{latin}{babel}
\usepackage{newcent}
\usepackage{babel}[ngerman,english,french]
\usepackage[T1]{fontenc}
\usepackage[utf8]{luainputenc}
\usepackage[portrait,a4paper]{geometry}
\geometry{verbose,tmargin=0.8cm,bmargin=0.8cm,lmargin=0.8cm,rmargin=0.8cm}
\usepackage{setspace}
\usepackage{multicol}
\usepackage{microtype}
\usepackage{tikz}
\raggedbottom
\raggedcolumns

\newcommand{\noun}[1]{\textsc{#1}}

%%% Document specific commands

\newcommand{\headline}{\section*{{\LARGE{}Pentagame}}}
\newcommand{\tocent}{}
\newcommand{\translator}{}
\newcommand{\layout}{\headline
\begin{onehalfspace}
    \general
\end{onehalfspace}
\noindent\hrulefill
\begin{multicols}{3}
    \choosex
       \columnbreak
    \setup
       \columnbreak
    \objective
\end{multicols}
\noindent\hrulefill
\begin{onehalfspace}
    \rules
\end{onehalfspace}
\vfill
\website
\newpage}
\newcommand{\general}{}
\newcommand{\choosex}{}
\newcommand{\setup}{}
\newcommand{\objective}{}
\newcommand{\rules}{}
\newcommand{\website}{
    \begin{flushright}
    \textsf{\textbf{pentagame.org}}
    \par\end{flushright}
}


\setlength{\parskip}{0ex}
\setlength{\parindent}{0cm}


%% Actual Document. Calling the content first, then printing the same layout.

\begin{document}

\selectlanguage{english}
\renewcommand{\headline}{\section*{{\LARGE{}Penta$\cdot$game} - English}}
\renewcommand{\tocent}{English}
\renewcommand{\general}{
A game that can be explained in a minute but remains fascinating
for years.

 Two players play in just 20 minutes. Three or four players
may take up to 90. 

There are no dice involved. It is fine for all people from
age 5. A game by Jan \noun{Suchanek.}

In the box: 4$\times$5 hand painted figures, 5 black and
5 gray blocks, the board.
}

\renewcommand{\choosex}{
\subsubsection*{Choose your figures}
Every player has five figures. There is one team of figures per player
in the box. One team has silver hair, one team has black hair, one
has golden hair and one is bald. 

Players should play the team they look like. 

You have a blue, a red, a white, a green and a yellow figure. 

They start at the five corners of the board matching their colour.

They want to reach the big stops in the middle.
}

\renewcommand{\setup}{
\subsubsection*{Setup}
Put your figures on the big corners at the rim matching their body
colours: your white figure on the white corner at the rim, your blue
figure on the blue corner at the rim, etc.

Put the black blocks on the five crossings in the middle of the board. 

Park the gray blocks in the centre for later. 
}

\renewcommand{\objective}{
\subsubsection*{Objective}
White figures want to go to the white crossing in the centre, blue
ones to the blue crossing, etc. The destination is always the big
coloured stop in the middle opposite the starting point.

Be the first to move \emph{three} pieces to their destinations to
win.
}

\renewcommand{\rules}{
\subsection*{The Rules}
Move any of your figures on the star or the rim in any direction as far as you please. 

You can turn at any free corner or crossing without stopping.

Never jump\textemdash neither over blocks, nor over any figures. 

\medskip

But you may move \emph{onto} a stop that is occupied:

\medskip

If you then beat a black block, place it on a free stop of your choice.

If you then move onto a stop with another figure, swap position with it.
 
\medskip

You may swap the positions of two of your own figures. 

If you move onto a stop with multiple figures, choose one to swap with.
 
\medskip 
 
Do not make the very same move twice in immediate succession.  

\medskip 

A figure that has reached its destination is removed. Put it into the centre of the board. Then take one of the gray blocks and put it on a free stop of your choice.

If you beat such a gray block, take it from the board again.

\medskip

Whoever moves \emph{three }of his figures out first wins.

\medskip

Excessive talking disqualifies.
}
\layout
\selectlanguage{ngerman}
\renewcommand{\headline}{\section*{{\LARGE{}Penta$\cdot$spiel} - Hochdeutsch}}
\renewcommand{\tocent}{Hochdeutsch}
\renewcommand{\general}{
Ein Spiel, das man in einer Minute begreift, das aber jahrelang fasziniert.

Zu zweit dauert die Partie nur 20 Minuten. Drei oder vier Spieler spielen 40-90 Minuten.

Es gibt keine Würfel. Geeignet für Spieler ab 5 Jahren. Ein Spiel von Jan  \noun{Suchanek.}

In der Schachtel: 4$\times$5 handbemalte Figuren, 5 schwarze und 5 graue Blöcke und das Spielbrett.
}

\renewcommand{\choosex}{
\subsubsection*{Figuren}
Jeder Spieler führt eine Mannschaft von fünf Figuren: da ist eine mit grauem, eine mit schwarzem, eine mit blondem und eine ohne Haar.

Jeder sucht sich eine Mann\-schaft aus.

Dann hat jeder eine blaue, eine rote, eine weiße, eine grüne und eine gelbe Figur.

Diese laufen alle von den fünf Ecken zu den fünf Kreuzungen.
}

\renewcommand{\setup}{
\subsubsection*{Aufbauen}
Stellt die Figuren geordnet je nach Farbe auf die fünf Ecken auf dem Kreis: die weißen auf das weiße Eckfeld, die blauen auf das blaue usw.

Je ein schwarzer Block kommt auf die Kreuzungen in der Mitte.

Die grauen Blöcke bleiben erst mal im Zentrum.
}

\renewcommand{\objective}{
\subsubsection*{Ziel des Spiels}
Weiße Figuren möchten zu der weißen Kreuzung, blaue zur blauen usw. 
Vom Rand her ist das Ziel immer die Kreuzung gleicher Farbe gegenüber.

Wer zuerst \emph{drei} seiner Figuren auf ihre jeweiligen Ziele zieht, gewinnt.
}

\renewcommand{\rules}{
\subsection*{Spielregeln}
Ziehe eine deiner Figuren auf Stern oder Kreis in beliebiger Richtung so weit du kannst.

Du kannst dabei an jeder freien Kreuzung ohne anzuhalten abbiegen.

Du darfst ziehen, aber nicht springen! Weder über Blöcke noch über andere Figuren!

\medskip

Jedoch kannst du auf ein Feld ziehen das besetzt ist und schlagen:

\medskip

Schlägst du einen schwarzen Block, so setze ihn auf ein beliebiges freies Feld. 

Schlägst du eine andere Spielfigur, so tauschen deine und diese Figur ihre Positionen.

\medskip

Auf diese Weise kann man zwei seiner eigenen Figuren vertauschen.

Ziehst du auf ein Feld, auf dem noch mehrere Figuren stehen, musst du mit \emph{einer }von ihnen tauschen.
 
\medskip 
 
Man darf nicht denselben Zug zweimal machen. 

\medskip 

Hat eine Figur ihr Ziel erreicht, so zieht sie raus. Sie kommt ins Zentrum. Dafür darf man von dort einen grauen Block ins Brett auf ein freies Feld setzen.

Schlägt man so einen grauen Block, so kommt er wieder raus.


\medskip

Wer zuerst \emph{drei }seiner Figuren herauszieht, gewinnt. 

\medskip

Wer zu viel quatscht wird disqualifiziert.
}

\layout
\selectlanguage{ngerman}
\renewcommand{\headline}{\section*{{\LARGE{}Penta$\cdot$spiel} - Flachdeutsch}}
\renewcommand{\tocent}{Flachdeutsch}
\renewcommand{\translator}{J.S.}
\renewcommand{\general}{
N geiles Spiel, dasde inner Minute kapiern und jahrelang spieln kannst.

Zu zweit geht echt schnell, drei oder vier spieln länger.

Würfel gibts nich. Passt für alle ab 5. Von Jan.

In der Box sind 4 mal 5 Figuren, dann noch 5 schwarze und 5 graue Blöcke und das Brett.
}

\renewcommand{\choosext}{Figuren}
\renewcommand{\choosex}{
Jeder hat ne Crew von fünf Figuren. Eine Crew hat graue Haare, eine hat schwarze usw., davon sucht sich jeder eine aus.

Dann hat jeder ne Figur von jeder Farbe.

Alle laufen von den Ecken zu den bunten Kreuzungen.
}

\renewcommand{\setup}{Aufbauen}
\renewcommand{\setup}{
Stellt eure Leute je nach Farbe auf die fünf großen bunten Ecken am Rand: die weißen auf die weiße, die blauen auf die blaue usw.

Auf jede Kreuzung in der Mitte tustn schwarzen Block.

Die grauen Blöcke parkste erstmal im Zentrum.
}

\renewcommand{\objectivet}{Worums geht}
\renewcommand{\objective}{
Weiße wolln zur weißen Kreuzung, blaue zur blauen usw. Also das Ziel ist immer gegenüber. 

Wer das zuerst mit \emph{drei} Figuren hinkriegt, gewinnt.
}

\renewcommand{\rulest}{Regeln}
\renewcommand{\rules}{
Du kannst laufen so weit und wolang du willst, bloß der Weg muss frei sein. 

Wenn frei ist kannste sogar abbiegen.

Aber überholen oder überspringen is nich!

\medskip

Schlagen kannste aber, und zwar so:

\medskip

N schwarzen Block kannste schlagen, dann tustn woanders hin.

Ne andere Spielfigur kannste auch schlagen, dann tauschste die zwei hin und her.

\medskip

So kannste auch zwei deiner eigenen Figuren swappen. 

Ziehste aufn Feld wo noch mehrere stehen, dann tauschste nur mit einer.

 
\medskip 
 
Genau dasselbe zweimal machen geht nich. 

\medskip 

Wennde ankommst, ziehste raus. Dafür kriechste n grauen Block, den kannste hintun wo du willst.

Schlägste so nen grauen Block so kommter wieder raus.


\medskip

Wer zuerst \emph{drei }rauszieht gewinnt. 

\medskip

Schnauze halten!
}

\layout

\selectlanguage{french}
\renewcommand{\headline}{\section*{{\LARGE{}Penta$\cdot$jeux} - Fran\c{c}ais}}
\renewcommand{\tocent}{Fran\c{c}ais}
\renewcommand{\translator}{Armelle Journaix}
\renewcommand{\general}{ 
Un jeu de société qui s’explique en une minute mais reste fascinant pendant des années.

À deux, la partie dure environ 20 minutes. 
À trois ou quatre, la partie peut durer jusqu'à 90 minutes.

Aucun dé requis.
Âge : à partir de 5 ans. 
Par Jan \noun{Suchanek}.

Contenu de la boîte : 4 familles de 5 pions en couleurs, 5 pions noirs, 5 pions gris, un plateau.

}
\renewcommand{\choosext}{Choisissez vos pions}
\renewcommand{\choosex}{
Chaque joueur dispose de cinq pions.
Une famille est composée de cinq pions et il y en a cinq par joueur dans la boîte.

Il y a la famille aux cheveux argentés, la famille aux cheveux noirs, une famille aux cheveux dorés et une famille chauve.

Chaque équipe est composée de cinq pions : un bleu, un rouge, un blanc, un vert et un jaune.
}

\renewcommand{\setupt}{Installation}
\renewcommand{\setup}{ 
Disposez vos pions sur les cercles extérieurs correspondant aux couleurs des pions : votre pion blanc sur le cercle extérieur blanc, votre pion bleu sur le cercle extérieur bleu, etc.

Disposez les pions noirs sur les cercles intérieurs en couleurs.

Posez les pions gris au centre du plateau de jeu.
}

\renewcommand{\objectivet}{But du jeu}
\renewcommand{\objective}{
Les pions blancs doivent se rendre sur le cercle intérieur blanc, les pions bleus sur le cercle intérieur bleu, etc.

Le but est toujours le cercle intérieur coloré à l’opposé du cercle de départ.

Soyez le premier à déplacer trois de vos cinq pions sur leurs cercles intérieurs colorés respectifs pour gagner.
}

\renewcommand{\rulest}{Règles du jeu}
\renewcommand{\rules}{
Déplacez le pion de votre choix sur le plateau de jeu, où vous voulez et dans n'importe quelle direction.

Vous pouvez utiliser tous les chemins possibles.

Vous ne pouvez pas sauter les cercles occupés par des pions, quelle que soit leur couleur.

\medskip

Mais vous pouvez poser votre pion sur un cercle déjà occupé :

\medskip

Si vous posez votre pion sur un cercle occupé par un pion noir, prenez le pion noir et posez-le ensuite où vous voulez sur un cercle libre.

Si vous déplacez votre pion sur un cercle occupé par un autre pion, vous changez de place avec lui.

\medskip

Vous pouvez changer de place entre deux pions de votre famille.

Si vous déplacez votre pion sur une case occupée par plusieurs autres pions, choisissez-en un et changez de place avec lui.

\medskip

Vous ne pouvez pas effectuer le même déplacement deux fois de suite.

\medskip

Un pion qui atteint son but est retiré du jeu.

Ensuite, prenez l‘un des pions gris du centre du jeu et placez-le sur le cercle de votre choix.

Si votre pion se pose à la place d’un pion gris, le pion gris est retiré du jeu.

\medskip

Le premier joueur qui retire trois de ses pions du plateau gagne la partie.

\medskip

Parler excessivement vous disqualifie.
}
\layout

\selectlanguage{latin}
\renewcommand{\headline}{\section*{{\LARGE{}Penta$\cdot$ludus} - Lingua latine}}
\renewcommand{\tocent}{Lingua latine}
\renewcommand{\translator}{Dr.~J\"org Grimm / J.S.}
\renewcommand{\general}{ 

\textbf{Descriptio }
Tabula pentagrammiforma, quattuor cohortes quinorum peditum coloratorum, quina impedimenta nigra et cana Pentaludum componunt. Fit Ioh.~\noun{Aridus (Suchanek).}

\textbf{Cohortes }
Lusor quisque agit pedites quinos, quibus est eadem forma (aut stella aut luna \&c.), sed sunt diversi colores. 

Sic tua cohors peditem caeruleum, peditem rubrum continet \&c.

\textbf{Disposito }
In quinque nodis positis in circulo, qui circumvenit pentagramma, ponite pedites eiusdem coloris: pedites albos in nodo albo, caeruleos
in nodo caeruleo \&c. 

Praeter illa decem loca quae nodos formant, existunt alter octoginta aedicula in viis, ubi pedites in ludo etiam possunt collocari. 

Sunt etiam impedimenta nigri et impedimenta cana. 

Impedimenta nigra ponite in nodis centri. In centro retinete impedimenta cana. 

\textbf{Propositum }

Destinatio per unum quemque peditem, qui occupat nodum externi circuli, est nodus eiusdem coloris in interno pentagrammatis. 
Albo est eundum ad album, rubro ad rubrum \&c. 

Lusor victor qui primus duxit tres pedites ad nodum aequi coloris.


}
\renewcommand{\choosext}{}
\renewcommand{\choosex}{ 
}

\renewcommand{\setupt}{} 
\renewcommand{\setup}{ 
}

\renewcommand{\objectivet}{} 
\renewcommand{\objective}{ 
}

\renewcommand{\rulest}{Regulae } 
\renewcommand{\rules}{ 
I. Duc peditum tuorum unum in quamvis directionem sequens aut viam circuli aut vias stellae, quoad per regulas potes occupare aediculum. 

II. In quodlibet aediculo libero, ubi  duae lineae conveniunt, tibi licet de via recta declinare sine retentione. 

III. At numquam tibi licet transcendere nec impedimenta nec pedites alios.

IV. Potes vero ingredi in aediculum occupatum:

a. Se ibi stat impedimentum nigrum, id pone in aediculo libero ad libitum. 

b. Se ibi iam stat pedes alius, pedites positiones suas mutare debent. Sed nota regulam V.

c. Se ibi sunt plures pedites (non potest fieri, nisi initio ludi), elige unum mutandum.

d. Ita potes etiam mutare positiones duorum peditum tuorum.

V. Eundem motum non licet iterum: bis idem motum ne cana.

VI. Pedes progressus ad finem ludo abit. Pone eum in centrum.
Accipe per illo impedimentum canum et colloca in aediculo, quoad tibi paret. 

VII. Quando captas tale impedimentum canum, remove a tabula.

VIII. Lusor  qui primus tres pedites ad finem duxit ludo victor.

IX. Et quietam! Lusores tacento.
}
\layout

\end{document}
