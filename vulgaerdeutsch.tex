\selectlanguage{ngerman}
\renewcommand{\headline}{Penta$\cdot$spiel - Einfaches Deutsch}
\renewcommand{\tocent}{Einfaches Deutsch}
\renewcommand{\translator}{J.S.}
\renewcommand{\general}{
N geiles Spiel, dasde inner Minute kapiern und jahrelang spieln kannst.

Zu zweit geht echt schnell, drei oder vier spielen länger.

Würfel gibt's nich. Passt für alle ab 5. Von Jan.

In der Box sind 4 mal 5 Figuren, 5 schwarze und 5 graue Blöcke und das Brett.
}

\renewcommand{\choosext}{Figuren}
\renewcommand{\choosex}{
Jeder hat 'ne Crew von fünf Figuren. 
%Eine Crew hat graue Haare, eine hat schwarze usw., davon sucht sich jeder eine aus.

Dann hat jeder von jeder Farbe 'ne Figur.

Alle laufen von den Ecken zu den bunten Kreuzungen.
}

\renewcommand{\setup}{Aufbauen}
\renewcommand{\setup}{
Stellt eure Leute je nach Farbe auf die fünf großen bunten Ecken am Rand: die weißen auf weiß, die blauen auf blaue und so weiter.

Auf jede Kreuzung in der Mitte tut 'nen schwarzen Block.

Die grauen Blöcke parkt ihr im Zentrum.
}

\renewcommand{\objectivet}{Worums geht}
\renewcommand{\objective}{
Weiße wollen zur weißen Kreuzung, blaue zur blauen und so weiter. Also ist das Ziel immer gegenüber. 

Wer das zuerst mit \emph{drei} Figuren hinkriegt, gewinnt.
}

\renewcommand{\rulest}{Regeln}
\renewcommand{\rules}{
Du laufen kannste so weit und wolang du willst, bloß der Weg muss frei sein. 

Wenn frei ist kannste sogar abbiegen.

Aber überholen oder überspringen is' nich'!

\myskip

Schlagen kannste aber, und zwar so:

\myskip

'nen schwarzen Block kannste schlagen, dann tust'n woanders hin.

'ne andere Spielfigur kannste auch schlagen, dann tauschste die zwei hin und her.

\myskip

So kannste auch zwei deiner eigenen Figuren vertauschen. 

Ziehste auf 'n Feld wo noch mehrere stehen, dann tauschst du nur mit einer.

\myskip 
 
Genau dasselbe zweimal machen geht nicht. 

\myskip 

Wenn du ankommst, ziehst du raus. Dafür kriegeste 'n grauen Block. Den kannste hintun wo du willst.

Schlägste so'nen grauen Block, so kommt er wieder raus.


\myskip

Wer zuerst \emph{drei }rauszieht gewinnt. 

\myskip

Und laber nicht zu viel!
}
