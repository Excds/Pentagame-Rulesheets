\selectlanguage{ngerman}
\renewcommand{\headline}{\section*{{\LARGE{}Penta$\cdot$spiel} - Vulgärdeutsch}}
\renewcommand{\general}{
N geiles Spiel, dasde inner Minute kapiern und jahrelang spieln kannst.

Zu zweit dauerts nur 20 Minuten. Drei oder vier spieln 40-90 Minuten.

Würfel gibts nich. Gut für alle ab 5. Von Jan  \noun{Suchanek.}

In der Box: 4$\times$5 Figuren, 5 schwarze und 5 graue Blöcke und das Brett.
}

\renewcommand{\choosex}{
\subsubsection*{Figuren}
Jeder hat ne Mannschaft mit fünf Figuren: da is eine mit grauen Haaren, eine mit schwarzen usw.

Jeder sucht sich eine Mann\-schaft aus.

Dann hat jeder ne blaue, ne rote, ne weiße, ne grüne und ne gelbe Figur.

Die laufen alle von den fünf Ecken zu den fünf Kreuzungen.
}

\renewcommand{\setup}{
\subsubsection*{Aufbauen}
Stellt euern Krempel je nach Farbe auf die fünf Ecken auf dem Kreis: die weißen auf das weiße Eckfeld, die blauen auf das blaue usw.

Auf jede Kreuzung in der Mitte tuste einen schwarzen Block.

Die grauen Blöcke parkste erst mal im Zentrum.
}

\renewcommand{\objective}{
\subsubsection*{Worums geht}
Weiße möchten zur weißen Kreuzung, blaue zur blauen usw. Also das Ziel ist immer gegenüber. 

Wer das zuerst mit \emph{drei} seiner Figuren schafft, gewinnt.
}

\renewcommand{\rules}{
\subsection*{Spielregeln}
Lauf auf Stern oder Kreis wolang und wie weit du willst.

Wenn frei ist kannste sogar abbiegen.

Aber über was rüber hüpfen is nich!

Schlagen kannst du aber, und zwar so:

\medskip

Schwarzen Block kannste schlagen, dann tuste ihn woanders hin.

Andere Spielfigur kannste ooch schlagen, dann tauschste mit ihr die Position.

\medskip

So kannste auch zwei deiner eigenen Figuren tauschen. 

Ziehste aufn Feld wo noch mehrere stehen, dann tauschte nur mit einer.

 
\medskip 
 
Genau dasselbe zweimal geht nich. 

\medskip 

Wenn du ankommst, ziehste raus. Dafür kriechste n grauen Block, den kannste hintun wo du willst.

Schlägste nen grauen Block so geht er wieder weg.


\medskip

Wer zuerst \emph{drei } rauszieht gewinnt. 

\medskip

Wer zu viel quatscht wird disqualifiziert.
}
